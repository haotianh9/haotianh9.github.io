\documentclass[10pt,letterpaper]{res}
\usepackage{color,soul}
%\usepackage[latin1]{inputenc}
\usepackage{amsmath,amsfonts,amssymb}
\usepackage{array,longtable}
\usepackage{datetime}
\usepackage{fontawesome}
\usepackage{academicons}
\usepackage[left=.5in,right=1.25in,top=.5in,bottom=.5in]{geometry}

%xelatex parameter
% \usepackage{xltxtra,fontspec,xunicode}
% \defaultfontfeatures{Scale=MatchLowercase}
% %\setromanfont{Adobe Garamond Pro}
% \setromanfont{Garamond}
%xelatex parameter

%\usepackage{helvetica} % uses helvetica postscript font (download helvetica.sty)
%\usepackage{newcent}   % uses new century schoolbook postscript font

\newcommand{\tab}{\hspace*{2em}}
\newcommand{\vsrk}{\vspace*{-12pt}}
\newcommand{\sepline}{\smallskip\moveleft\hoffset\vbox{\hrule width 7.5in height .5pt}
	\vspace*{-5pt}}

\newdateformat{monthyear}{%
	\monthname[\THEMONTH], \THEYEAR}

\usepackage{hyperref}
%\usepackage[compact]{titlesec}
%\titlespacing{\section}{0pt}{*0}{*0}
%\titlespacing{\subsection}{0pt}{*0}{*0}
%\titlespacing{\subsubsection}{0pt}{*0}{*0}

\catcode`@=11

\def\nvphantom{\v@true\h@false\nph@nt}
\def\nhphantom{\v@false\h@true\nph@nt}
\def\nphantom{\v@true\h@true\nph@nt}
\def\nph@nt{\ifmmode\def\next{\mathpalette\nmathph@nt}%
  \else\let\next\nmakeph@nt\fi\next}
\def\nmakeph@nt#1{\setbox\z@\hbox{#1}\nfinph@nt}
\def\nmathph@nt#1#2{\setbox\z@\hbox{$\m@th#1{#2}$}\nfinph@nt}
\def\nfinph@nt{\setbox\tw@\null
  \ifv@ \ht\tw@\ht\z@ \dp\tw@\dp\z@\fi
  \ifh@ \wd\tw@-\wd\z@\fi \box\tw@}

\begin{document}
	\moveleft.5\hoffset\centerline{\huge\bf HAOTIAN HANG}
	\smallskip
	\moveleft.5\hoffset\centerline{\monthyear\today}
	\begin{resume}%
		\vsrk
		% \section{PERSONAL INFO}
		% \sepline
		% \begin{longtable}{@{}>{\raggedright}p{0.5\linewidth}
		% 		p{\dimexpr0.5\linewidth-2\tabcolsep-\arrayrulewidth\relax}@{}}
		% 	{\bf Birth Year:} 1997 & {\bf Address:} 1247 W 30 St, Los Angeles, CA 90007\\
		% 	{\bf Citizenship:} People's Republic of China & {\bf Mobile:} (213)462-5919\\
		% 	{\bf E-mail:} haotianh@usc.edu 
		% 	& {\bf Webpage:}   \href{https://haotianh9.github.io/}{\textit{Personal Website}}
		% 	\\
		% 	\multicolumn{2}{l}{\hspace{-2.25ex} {\bf Social media:}\href{https://github.com/haotianh9}{\textit{Github}} \href{https://www.linkedin.com/in/haotian-hang-0a23471a3/}{\textit{LinkedIn}} \href{https://www.researchgate.net/profile/Haotian-Hang/}{\textit{ResearchGate}} \href{https://scholar.google.com/citations?user=3hzPBNoAAAAJ}{\textit{Google Scholar}}} \\
		% \end{longtable}
		% \vsrk\vsrk
		\section{Contact Information}
		\sepline
		\begin{longtable}{@{}>{\raggedright}p{0.5\linewidth}
				p{\dimexpr0.5\linewidth-2\tabcolsep-\arrayrulewidth\relax}@{}}
			{\bf Mobile:} (213)462-5919 & {\bf Address:} 920 Downey way, Los Angeles, CA 90089\\
			{\bf E-mail:} haotianh@usc.edu, hanghaotian@gmail.com
			& {\bf \faHome:}   \href{https://haotianh9.github.io/}{\textit{Personal Website }}
			\\
			\multicolumn{2}{l}{\hspace{-2.25ex} {\bf Social media:} \hspace{1ex}  \href{https://github.com/haotianh9}{\textit{\faGithub}} \hspace{1ex} \href{https://www.linkedin.com/in/haotian-hang-0a23471a3/}{\textit{\faLinkedin}} \hspace{1ex} \href{https://www.researchgate.net/profile/Haotian-Hang/}{\textit{\aiResearchGate}} \hspace{1ex} \href{https://scholar.google.com/citations?user=3hzPBNoAAAAJ}{\textit{\aiGoogleScholar}}} \hspace{1ex} \\
		\end{longtable}
		\vsrk\vsrk
		\section{Education}
		\sepline
		\begin{longtable}{@{}>{\raggedright}p{0.07\linewidth}
				p{\dimexpr0.95\linewidth-2\tabcolsep-\arrayrulewidth\relax}@{}}
			
			\hspace{-4ex}{2020 -  2025} & {\bf University of Southern California}, Los Angeles, CA\\
			& {~Doctor of Philosophy, Mechanical Engineering, (defense date: 2025/3/14)} \\
			% & {~Master of Science, Mechanical Engineering, December 2021} \\
			& {~Master of Science, Computer Science (High Performance Computing track),  Dec 2023} \\

			\hspace{-4ex}{2015 - 2019} & {\bf Shanghai Jiao Tong University}, Shanghai, China\\
			& {~B.S. Aeronautics and Astronautics Engineering},  June 2019 (Average Score: 89.22/100)\\[-1ex]
		\end{longtable}
		\vsrk\vsrk
		\section{Employment}
		\sepline
		\begin{longtable}{@{}>{\raggedright}p{0.07\linewidth}
				p{\dimexpr0.95\linewidth-2\tabcolsep-\arrayrulewidth\relax}@{}}
    	\hspace{-4ex}{2020 -  } & {\bf Research Assistant}, Bio-Inspired Motion Lab at USC, PI: \textit{Prof.~Eva Kanso}\\
				\hspace{-4ex}{2023/6 - 8} & {\bf Quantitative Analyst Intern}, Corporate Model Risk, Wells Fargo, Manager: \textit{Dr.~Nengfeng Zhou}, \textit{Dr.~Harry Zhang}\\
				\hspace{-4ex}{ } & Evaluate robustness of machine learning models \\
            \hspace{-4ex}{ } & \hspace{4ex} Discover the contribution of each input feature on robustness metrics and overfitting \\
            \hspace{-4ex}{ } &  \hspace{4ex} Compare different perturbation schemes in robustness testing \\
            \hspace{-4ex}{ } &  \hspace{4ex} Develop nonlinear variance inflation factor (VIF) to evaluate the nonlinear correlation in dataset \\
			\hspace{-4ex}{2019/9 - 12} & {\bf Intern Algorithmic Engineer}, Shanghai Hongpu Information Technology Co., Ltd. \\
			\hspace{-4ex}{ } & Conduct flaw detection on images of photovoltaic cell using Faster R-CNN and yolov3 \\
			\hspace{-4ex}{2016 - 2019} & {\bf Research Assistant}, J.C.Wu Center for Aerodynamics, PI: \textit{Prof.~Hong Liu}\\
		\end{longtable}
		\vsrk\vsrk
		\section{Publications}
		\sepline
		\begin{longtable}{@{}>{\raggedright}p{0.07\linewidth}
				p{\dimexpr0.95\linewidth-2\tabcolsep-\arrayrulewidth\relax}@{}}
                % all published papers are in APA format in google scholar!!!
                \hspace{-4ex}{2025}
                 & \nhphantom{13. }13. Cheng, H., \textbf{Hang, H.}, Huang, C., Barnett, A., \& Kanso, E.$^*$ (in preparation) Collective transitions in bi-chamber domain \\
                &  \nhphantom{12. }12.  \textbf{Hang, H.}$^\#$, Shen, Y.$^\#$, Zhu, V.$^{\#,*}$,   Li, M., Cruz, J.  (in preparation) Chitchat with AI: understand the supply chain carbon disclosure of companies worldwide through Large Language Model  \\
                 & \nhphantom{11. }11. \textbf{Hang, H.}, Huang, C., Barnett, A., \& Kanso, E.$^*$ (in preparation) Chaotic Mixing in Large Schools of Fish \\
                & \nhphantom{10. }10. Linot, A.$^*$, \textbf{Hang, H.},  Kanso, E.,  Taira, K.  (under review)
\href{https://arxiv.org/abs/2501.00626}{\textit{Hierarchical equivariant graph neural networks for forecasting collective motion in vortex clusters and microswimmers}} \\
                 & \nhphantom{9. }9. \textbf{Hang, H.}, Huang, C., Barnett, A., \& Kanso, E.$^*$ (under review) \href{https://arxiv.org/abs/2505.05822 }{Self-reorganization and Information Transfer in Massive Schools of Fish} \\
                       & \nhphantom{8. }8. \textbf{Hang, H.}, Jiao, Y., Merel, J. \& Kanso, E.$^*$ (under review). \href{https://www.biorxiv.org/content/10.1101/2023.12.15.571932v2}{\textit{Flow Currents Support Simple and Versatile Trail-Tracking Strategies}}  \\
                    & \nhphantom{7. }7.  Jiao, Y.$^\#$, \textbf{Hang, H.}$^\#$,  Merel, J., \& Kanso, E.$^*$ (2025). \href{https://doi.org/10.1038/s41467-025-58125-6}{Sensing flow gradients is necessary for learning autonomous underwater navigation}. Nature Communications, 16(1), 3044.
                    \\
    \hspace{-4ex}{2024}   & \nhphantom{6. }6. 
        Heydari, S.$^\#$, \textbf{Hang, H.}$^\#$, \& Kanso, E.$^*$ (2024). \href{https://doi.org/10.7554/eLife.96129.1}{\textit{Mapping spatial patterns to energetic benefits in groups of flow-coupled swimmers}}. Elife, 13, RP96129. \\
    & \nhphantom{5. }5. \textbf{Hang, H.},  Heydari, S. \& Kanso, E.$^*$   (2024). \href{http://dx.doi.org/10.23919/ACC60939.2024.10644662}{\textit{Feedback control of uncoordinated flapping swimmers to maintain school cohesion}}. American Control Conference (ACC)  \\
		  \hspace{-4ex}{2023}
		   & \nhphantom{4. }4. Qin, S., \textbf{Hang, H.}, Xiang, Y.$^*$ \& Liu, H. (2023). \href{https://doi.org/10.1016/j.cja.2023.09.030}{\textit{Reynolds-number scaling analysis on lift generation of a flapping and passive rotating wing with an inhomogeneous mass distribution}}.  Chinese Journal of Aeronautics, 37(2), 259-269  \\
			\hspace{-4ex}{2022} 
			& \nhphantom{3. }3. \textbf{Hang, H.}, Heydari, S., Costello, J., \& Kanso, E.$^*$ (2022). \href{https://doi.org/doi:10.1017/jfm.2021.984}{\textit{Active tail flexion in concert with passive hydrodynamic forces improves swimming speed and efficiency}}. Journal of Fluid Mechanics, 932, A35.  \\
			\hspace{-4ex}{2021} 
			& \nhphantom{2. }2. Xiang, Y., \textbf{Hang, H.}, Qin, S.$^*$, \& Liu, H. (2021). \href{https://doi.org/10.1007/s10409-021-01134-7}{\textit{Scaling analysis of the circulation growth of leading-edge vortex in flapping flight}}. Acta Mech. Sin, 37(10), 1530-1543.  \\
			\hspace{-4ex}{2020} & \nhphantom{1. }1. \textbf{Hang, H.}, Yu, B., Xiang, Y., Zhang, B.$^*$, \& Liu, H. (2020).  \href{https://doi.org/10.1007/s12650-019-00605-1}{\textit{An objective-adaptive refinement criterion based on modified ridge extraction method for finite-time Lyapunov exponent (FTLE) calculation}}. Journal of Visualization, 23(1), 81-95. \\
            & $^\#$ - equal contribution, $^*$ - corresponding author \\
		\end{longtable}
		\vsrk\vsrk
		\section{Patents}
		\sepline
		\begin{longtable}{@{}>{\raggedright}p{0.07\linewidth}
				p{\dimexpr0.95\linewidth-2\tabcolsep-\arrayrulewidth\relax}@{}}
    	\hspace{-4ex}{2018}    & Single-wing aircraft with rotor wing mode and fixed wing flight mode and mode switching method ZL.201811222322.5\\
		\end{longtable}
		\vsrk\vsrk	
  \section{Talks/Presentations}
		\sepline
		\begin{longtable}{@{}>{\raggedright}p{0.07\linewidth}
				p{\dimexpr0.95\linewidth-2\tabcolsep-\arrayrulewidth\relax}@{}}
                \hspace{-4ex}{2025}     & {\bf APS March Meeting}, Fish schooling at extreme scales  \\
                & {\bf Dynamics Day}, Fish schooling at extreme scales  \\
    	\hspace{-4ex}{2024}     & {\bf APS Division of Fluid Dynamics Meeting}, Fish schooling at extreme scales \\
        & {\bf APS March Meeting}, Learning to track flows\\
				\hspace{-4ex}{2023}
     & {\bf APS Division of Fluid Dynamics Meeting}, Flow-coupled swimmers self-organize into energetically cooperative or greedy spatial patterns\\
    & {\bf So Cal Fluids XVI}, Active tail flexion in concert with passive hydrodynamic forces improves swimming speed and efficiency \\
			\hspace{-4ex}{2022} 
			& {\bf APS Division of Fluid Dynamics Meeting}, Learning to blindly follow hydrodynamic trails\\
			& {\bf So Cal Fluids XV}, Learning to blindly follow hydrodynamic trails \\
			\hspace{-4ex}{2021} & {\bf APS Division of Fluid Dynamics Meeting}, Active tail flexion in concert with passive hydrodynamic forces improves swimming speed and efficiency\\
			\hspace{-4ex}{2020} & {\bf APS Division of Fluid Dynamics Meeting}, Flowtaxis in the wakes of oscillating airfoils\\
			\hspace{-4ex}{2018} & {\bf APS Division of Fluid Dynamics Meeting}, Passive rotation of a flapping wing with an inhomogeneous mass distribution \\

		\end{longtable}
		\vsrk\vsrk

		\section{Research Interests/Experience }
		\sepline
		\begin{longtable}{@{}>{\raggedright}p{0.07\linewidth}
				p{\dimexpr0.95\linewidth-2\tabcolsep-\arrayrulewidth\relax}@{}}
           \hspace{-4ex}{2023 - } & {\bf Fish schooling at extreme scales and in complex geometry}, supervised by \textit{Prof.~Eva Kanso}, \textit{Prof.~Alex Barnett} \\
           & Employ Boundary Element method (BEM) to model the hydrodynamic interaction between fish school and arbitrary geometry \\ 
           % & Study the transition of fish school in a bi-chamber domain \\
         & Scale agent-based simulation to the order of $10^4$  and analyze the data based on continuum description  \\
			\hspace{-4ex}{2021 - } & {\bf School cohesion and energetic benefits of fish school}, supervised by \textit{Prof.~Eva Kanso}, \textit{Prof.~Matt McHenry} \\
			& Develop efficient parallelized code using  fast multipole method (FMM) to simulate emergent formation of hydordynamically-coupled siwmmers\\
			& Evaluate the energetic benefit and stability of fish schools of different different spatial patterns \\
			& Design control laws to stabilize fish schools that are passively unstable \\
            & Study the dynamically-changing real fish schools using graph neural network\\
			\hspace{-4ex}{2020 - } & {\bf Underwater navigation using deep reinforcement learning}, supervised by \textit{Prof.~Eva Kanso}, \textit{Dr.~Josh Merel}  \\
			& Employ reinforcement learning to track vortical wakes based on local flow sensory and to navigate through unsteady adversarial background flow\\
			& Find the importance of the wake's periodicity and traveling wave characteristic in both tasks \\
			& Analyze the wake-trac king policy in a simplified signal field and prove that stability of the controller depends on the location of sensor \\
			& Explain neural-network-based policy in observation space, and link it to the generalizability of the policy \\
		\end{longtable}
		\vspace*{-4.5ex}
		\begin{longtable}{@{}>{\raggedright}p{0.07\linewidth}
				p{\dimexpr0.95\linewidth-2\tabcolsep-\arrayrulewidth\relax}@{}}
			
			\hspace{-4ex}{2020 - } & {\bf Fluid-structure interaction of biologically-inspired flexible propulsor}, supervised by \textit{Prof.~Eva Kanso}, \textit{Prof.~John H. Costello}\\
			& Analyze the role of active and passive flexion on swimming speed and efficiency of a self-propelling pitching plate using vortex sheet method \\
			& Parametric study on effects of flexion phase, flexion angle and flexion ratio on swimming performance\\
			& Find overlap between biological data and the region we proposed to have hydrodynamic benefits in parameter space\\
            
						
		\end{longtable}
		\vspace*{-4.5ex}
		\begin{longtable}{@{}>{\raggedright}p{0.07\linewidth}
				p{\dimexpr0.95\linewidth-2\tabcolsep-\arrayrulewidth\relax}@{}}
			\hspace{-4ex}{2016 - 2019} & {\bf  High lift generation mechanisms of insects' flight}, supervised by \textit{Prof.~Hong Liu}, \textit{Prof.~Yang Xiang} and \textit{Dr.~Suyang Qin}\\
			& Conduct experimental study using robotic flapping wing models in glycerin with Reynolds number similar to insects \\
			& Measure flow field using particle image velocimetry (PIV) and measure force and torque using 6-axis force sensor, analogue filter and NI data acquisition system \\
			& Study formation of leading edge vortex(LEV) for different kinematic modes, and find advanced rotation can generate a lager LEV because of wake capture \\
			& Find a scaling law between passive rotation and active translation in flapping wing model \\

		\end{longtable}
		\vspace*{-4.5ex}
		\begin{longtable}{@{}>{\raggedright}p{0.07\linewidth}
				p{\dimexpr0.95\linewidth-2\tabcolsep-\arrayrulewidth\relax}@{}}
			
			\hspace{-4ex}{2016 - 2019} & {\bf AMR for FTLE calculation}, supervised by \textit{Prof.~Hong Liu}, \textit{Prof.~Bin Zhang}, \textit{Bin Yu} and \textit{Prof.~Yang Xiang}\\
			& Develop a physics-based adaptive refinement method for finite-time Lyapunov exponent calculation\\
			
		\end{longtable}
		\vspace*{-4.5ex}
		\begin{longtable}{@{}>{\raggedright}p{0.07\linewidth}
				p{\dimexpr0.95\linewidth-2\tabcolsep-\arrayrulewidth\relax}@{}}
			
			\hspace{-4ex}{2015 - 2016} & {\bf VTOL pitch-changed quadrotor }, supported by National Students’ Platform for Innovation and Entrepreneurship Training, supervised by \textit{Prof.~Junqi Wu}\\
			& Lead a team to make a quadrotor and implement pitch-changed technique and VTOL technique in terms of both mechanical and control\\
			% & Include flight test and numerical simulation \\
		\end{longtable}
		
		\vsrk\vsrk
		
		\section{Teaching Experience}
		\sepline
		\begin{longtable}{@{}>{\raggedright}p{0.07\linewidth}
				p{\dimexpr0.95\linewidth-2\tabcolsep-\arrayrulewidth\relax}@{}}
			& \textbf{at University of Southern California} \\
			

			\hspace{-4ex}{2021 Spring} & {\bf Teaching Assistant}, AME-526, Introduction to mathematical methods in engineering II, \emph{Prof.~Niema Pahlevan}\\
	
			\hspace{-4ex}{2020 Fall} & {\bf Teaching Assistant}, AME-404, Computational Solutions to Engineering Problems, \emph{Prof.~Takahiro Sakai}\\
			
		\end{longtable}
		\vsrk\vsrk

			\section{Open Source Projects}
		\sepline
         \begin{longtable}{@{}>{\raggedright}p{0.07\linewidth}
				p{\dimexpr0.95\linewidth-2\tabcolsep-\arrayrulewidth\relax}@{}}
    \hspace{-4ex}{2023- 2024} &   Field-Oriented Control (FOC) on STM32  \href{https://github.com/haotianh9/DengFOC_on_STM32}{\textit{\faGithub} Github Link} 	\\
    \hspace{-4ex}{2021- 2024} &  Inferring unknown parameters of partially-observable system using Physics-informed-DeepONet  \href{https://github.com/chenchenhuang/DeepONet_physics_inferring}{\textit{\faGithub} Github Link} \\
     \hspace{-4ex}{2021 - 2022} &  Parallel C++ Implementation of Proximal Policy Optimization (PPO)  \href{https://github.com/haotianh9/Reinforcement_learning_cpp_parallel}{\textit{\faGithub} Github Link} \\
		\end{longtable}
        \vsrk\vsrk
        

\section{Student Mentees}
		\sepline
		\begin{longtable}{@{}>{\raggedright}p{0.07\linewidth}
				p{\dimexpr0.95\linewidth-2\tabcolsep-\arrayrulewidth\relax}@{}}
    & \textbf{at University of Southern California} \\
                \hspace{-4ex}{2023 - 2024} & Ziyan Zhu, M.S. student  \\
                & Ali Khokhar, M.S. student  \\
                \hspace{-4ex}{2023} & Donghun (Calvin) Moon, undergraduate at Columbia University \\
		\end{longtable}
		\vsrk\vsrk



		\section{Service}
		\sepline
		\begin{longtable}{@{}>{\raggedright}p{0.07\linewidth}
				p{\dimexpr0.95\linewidth-2\tabcolsep-\arrayrulewidth\relax}@{}}
                \hspace{-4ex}{2025} 
                 & Reviewer,  ICLR-AI4Mat 2025, Physics of Fluids, ICMI 2025  \\
                \hspace{-4ex}{2024} 
                 & Reviewer, Physics of Fluids, PNAS Nexus, NeurIPS-AI4Mat 2024  \\
                & Judge, Undergraduate Symposium for Scholarly and Creative Work \\
                & Session Chair, APS March Meeting DFD IX session \\
                \hspace{-4ex}{2023} & Reviewer, American Control Conference (ACC) 2024 \\
			       & Judge, Undergraduate Symposium for Scholarly and Creative Work \\
    \hspace{-4ex}{2022} & AME 441 project mentor, Robotic fish with artificial lateral line \\
			 & Judge, Undergraduate Symposium for Scholarly and Creative Work \\
		\end{longtable}
		\vsrk\vsrk

		\section{Honor/Awards}
		\sepline
		\begin{longtable}{@{}>{\raggedright}p{0.07\linewidth}
				p{\dimexpr0.95\linewidth-2\tabcolsep-\arrayrulewidth\relax}@{}}
                
	\hspace{-4ex}{2025}  & The William F. Ballhaus Jr. Prize for Excellence in Graduate Engineering Research (Viterbi School 2025 Best Dissertation Award)  \\
    & Dynamics Days US 2025 travel grant  \\
				\hspace{-4ex}{2022}  & USC Three Minute Thesis (3MT) competition Finalist \href{https://sites.google.com/usc.edu/eishub/three-minute-thesis/2022-3mt?authuser=0}{Link}\\
			\hspace{-4ex}{2020}  & USC Viterbi fellowship \\
			\hspace{-4ex}{2017-2018} & Hui-Chun Chin and Tsung-Dao Lee Chinese Undergraduate Research Endowment of SJTU \\
			\hspace{-4ex}{2016} & Honeywell Star Project  \\
			& {\bf Second Place }, Parts of the National College Students Physics Competition \\
			& {\bf Third Place }, Chinese College Students' Mathematics Competition \\
			\hspace{-4ex}{2014} & {\bf First Place}, Chinese Chemistry Olympiad \\
			 & {\bf First Place}, Shanghai Adolescents Science and Technology Innovation Contest \\

		\end{longtable}
		\vsrk\vsrk

		\section{Professional Society Memberships}
		\sepline
		\begin{longtable}{@{}>{\raggedright}p{0.07\linewidth}
				p{\dimexpr0.95\linewidth-2\tabcolsep-\arrayrulewidth\relax}@{}}
		\bullet	 &  American Physical Society (APS) \\
		\bullet		&  Institute of Electrical and Electronics Engineers (IEEE) \\
		\end{longtable}
		\vsrk\vsrk
  
		\section{Graduate Coursework}
		\sepline
		\begin{longtable}{@{}>{\raggedright}p{0.07\linewidth}
				p{\dimexpr0.95\linewidth-2\tabcolsep-\arrayrulewidth\relax}@{}}
			& \textbf{at University of Southern California} \\
   \hspace{-4ex}{2024} & AME-530b, Dynamics of Incompressible Fluids, \emph{Prof.~Mitul Luhar}\\
			\hspace{-4ex}{2023} & CSCI-575, Quantum Computing and Quantum Cryptography, \emph{Prof.~Ming-Deh Huang}\\
             &   CSCI-599, An Introduction to Programming Languages, \emph{Prof.~Mukund Raghothaman}\\
			\hspace{-4ex}{2022} & EE-587, Nonlinear Control Systems, \emph{Prof.~Mihailo Jovanovic} \\
			& CSCI-561, Foundations of Artificial Intelligence, \emph{Prof.~Wei-Min Shen} \\
			& CSCI-567, Machine Learning, \emph{Prof.~Victor Adamchik}\\
			& CSCI-653, High Performance Computing and Simulations, \emph{Prof.~Aiichiro Nakano}\\
			\hspace{-4ex}{2021} & PHYS-516, Methods of Computational Physics, \emph{Prof.~Aiichiro Nakano}\\
			& EE-556, Stochastic Systems and Reinforcement Learning, \emph{Prof.~Rahul Jain}\\\
			& CSCI-570, Analysis of Algorithms, \emph{Prof.~Victor Adamchik}\\
			& AME-508, Machine Learning and Computational Physics, \emph{Prof.~Assad Oberai}\\
			& CSCI-596, Scientific Computing and Visualization, \emph{Prof.~Aiichiro Nakano}\\
			\hspace{-4ex}{2020} & AME-525, Engineering Analysis, \emph{Prof.~Eva Kanso}\\
			& AME-526, Introduction to Mathematical Methods in Engineering II, \emph{Prof.~Niema Pahlevan}\\
			& AME-511, Compressible Gas Dynamics, \emph{Prof.~Iván Bermejo-Moreno}\\
			& PHYS-760, Selected Topics in Computational Physics, \emph{Prof.~Satish Kumar Thittamaranahalli}\\
			& AME-451, Linear Control Systems I, \emph{Prof.~Henryk Flashner}\\
			& AME-541, Linear Control Systems II, \emph{Prof.~Néstor O. Pérez-Arancibia}\\\
			& AME-535A, Introduction to Computational Fluid Mechanics, \emph{Prof.~Alejandra Uranga}\\
			& AME-530A, Dynamics of Incompressible Fluids, \emph{Prof.~Carlos Pantano}\\
			% & \textbf{at University of Southern California} \\
			% \hspace{-4ex}{2021} & PHYS-516, Methods of Computational Physics, A , \emph{Prof.~Aiichiro Nakano}\\
			% & EE-556, Stochastic Systems and Reinforcement Learning, A , \emph{Prof.~Rahul Jain}\\
			% & CSCI-570, Analysis of Algorithms, B+ , \emph{Prof.~Victor Adamchik}\\
			% & AME-508, Machine Learning and Computational Physics, A , \emph{Prof.~Assad Oberai}\\
			% & CSCI-596, Scientific Computing and Visualization, A , \emph{Prof.~Aiichiro Nakano}\\
			% & \hspace{4ex} \textbf{class project:} a C++ parallel reinforcement learning implementation \href{https://github.com/haotianh9/Reinforcement_learning_cpp_parallel}{\textit{Github Link}} \\
			% \hspace{-4ex}{2020} & AME-525, Engineering Analysis, A- ,\emph{Prof.~Eva Kanso}\\
			% & AME-526, Introduction to Mathematical Methods in Engineering II, A , \emph{Prof.~Niema Pahlevan}\\
			% & AME-511, Compressible Gas Dynamics, A , \emph{Prof.~Iván Bermejo-Moreno}\\
			% & PHYS-760, Selected Topics in Computational Physics, P , \emph{Prof.~Satish Kumar Thittamaranahalli}\\
			% & AME-451, Linear Control Systems I, A , \emph{Prof.~Henryk Flashner}\\
			% & AME-541, Linear Control Systems II, A- , \emph{Prof.~Néstor O. Pérez-Arancibia}\\
			% & AME-535A, Introduction to Computational Fluid Mechanics, A , \emph{Prof.~Alejandra Uranga}\\
			% & AME-530A, Dynamics of Incompressible Fluids, B+ , \emph{Prof.~Carlos Pantano}\\

		\end{longtable}
		\vsrk\vsrk

 
		\section{Online Coursework}
		\sepline
		\begin{longtable}{@{}>{\raggedright}p{0.07\linewidth}
				p{\dimexpr0.95\linewidth-2\tabcolsep-\arrayrulewidth\relax}@{}}
				% & \textbf{at Udacity} \\
    \hspace{-4ex}{2024} & Minds and Machines, MITx Online \\
			\hspace{-4ex}{2022} & C++ Nanodegree, Udacity \\
			& Qiskit Global Summer School 2022, IBM \\
				% & \textbf{at Coursera} \\
			\hspace{-4ex}{2021} & Build a Modern Computer from First Principles: From Nand to Tetris (Project-Centered Course), Coursera \\
			\hspace{-4ex}{2019} & {\bf Specialization}, DeepLearning.AI TensorFlow Developer , Coursera (containing 4 courses) \\
			& {\bf Specialization}, Deep Learning, Coursera (containing 5 courses) \\
			& Machine Learning, Coursera \\
			\hspace{-4ex}{2014} & General Chemistry, Coursera \\
		\end{longtable}
		\vsrk\vsrk
		
		\section{Technical Skills}
		\sepline
		\begin{longtable}{@{}>{\raggedright}p{0.07\linewidth}
				p{\dimexpr0.65\linewidth-2\tabcolsep-\arrayrulewidth\relax}@{}}
			\hspace{-24ex}{Programming Language: } & Python, \,  C/C++, \, Matlab, \, Fortran   \\
			\hspace{-24ex}{Machine learning framework: } & Pytorch, Tensorflow \\
			\hspace{-24ex}{Micro controller: } & Arduino, Raspberry Pi, Pixhawk, stm32 \\
			\hspace{-24ex}{Other softwares/ tools: } & Solidworks, Fusion 360,  ROS/ROS2, Gazebo, GitHub, \LaTeX, Docker, Ansys Fluent, Linux, MPI, OpenMP, cuda, KiCAD  \\
		\end{longtable}
		\vsrk\vsrk
	\end{resume} 
\end{document}