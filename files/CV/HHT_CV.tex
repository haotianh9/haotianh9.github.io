\documentclass[10pt,letterpaper]{res}
\usepackage{color,soul}
%\usepackage[latin1]{inputenc}
\usepackage{amsmath,amsfonts,amssymb}
\usepackage{array,longtable}
\usepackage{datetime}
\usepackage[left=.5in,right=1.25in,top=.5in,bottom=.5in]{geometry}

%xelatex parameter
% \usepackage{xltxtra,fontspec,xunicode}
% \defaultfontfeatures{Scale=MatchLowercase}
% %\setromanfont{Adobe Garamond Pro}
% \setromanfont{Garamond}
%xelatex parameter

%\usepackage{helvetica} % uses helvetica postscript font (download helvetica.sty)
%\usepackage{newcent}   % uses new century schoolbook postscript font

\newcommand{\tab}{\hspace*{2em}}
\newcommand{\vsrk}{\vspace*{-12pt}}
\newcommand{\sepline}{\smallskip\moveleft\hoffset\vbox{\hrule width 7.5in height .5pt}
	\vspace*{-5pt}}

\newdateformat{monthyear}{%
	\monthname[\THEMONTH], \THEYEAR}

\usepackage{hyperref}
%\usepackage[compact]{titlesec}
%\titlespacing{\section}{0pt}{*0}{*0}
%\titlespacing{\subsection}{0pt}{*0}{*0}
%\titlespacing{\subsubsection}{0pt}{*0}{*0}

\catcode`@=11

\def\nvphantom{\v@true\h@false\nph@nt}
\def\nhphantom{\v@false\h@true\nph@nt}
\def\nphantom{\v@true\h@true\nph@nt}
\def\nph@nt{\ifmmode\def\next{\mathpalette\nmathph@nt}%
  \else\let\next\nmakeph@nt\fi\next}
\def\nmakeph@nt#1{\setbox\z@\hbox{#1}\nfinph@nt}
\def\nmathph@nt#1#2{\setbox\z@\hbox{$\m@th#1{#2}$}\nfinph@nt}
\def\nfinph@nt{\setbox\tw@\null
  \ifv@ \ht\tw@\ht\z@ \dp\tw@\dp\z@\fi
  \ifh@ \wd\tw@-\wd\z@\fi \box\tw@}

\begin{document}
	\moveleft.5\hoffset\centerline{\huge\bf HAOTIAN HANG}
	\smallskip
	\moveleft.5\hoffset\centerline{\monthyear\today}
	\begin{resume}%
		\vsrk
		\section{PERSONAL INFO}
		\sepline
		\begin{longtable}{@{}>{\raggedright}p{0.5\linewidth}
				p{\dimexpr0.5\linewidth-2\tabcolsep-\arrayrulewidth\relax}@{}}
			{\bf Birth Year:} 1997 & {\bf Address:} 1247 W 30 St, Los Angeles, CA 90007\\
			{\bf Citizenship:} People's Republic of China & {\bf Mobile:} (213)462-5919\\
			{\bf E-mail:} haotianh@usc.edu 
			& {\bf Webpage:}   \href{https://haotianh9.github.io/}{\textit{Personal Website}}
			\\
			{\bf Social media:}
			\href{https://github.com/haotianh9}{\textit{Github}}
			\href{https://www.linkedin.com/in/haotian-hang-0a23471a3/}{\textit{LinkedIn}}
			\href{https://www.researchgate.net/profile/Haotian-Hang/}{\textit{ResearchGate}}
			\href{https://scholar.google.com/citations?hl=en&user=3hzPBNoAAAAJ&view_op=list_works&authuser=1&sortby=pubdate}{\textit{Google Scholar}}
		\end{longtable}
		\vsrk\vsrk
		\section{EDUCATION}
		\sepline
		\begin{longtable}{@{}>{\raggedright}p{0.07\linewidth}
				p{\dimexpr0.95\linewidth-2\tabcolsep-\arrayrulewidth\relax}@{}}
			
			\hspace{-4ex}{2020 - } & {\bf University of Southern California}, Los Angeles, CA\\
			& {~Ph.D. Student, Mechanical Engineering} \\
			& {~Master of Science, Mechanical Engineering, December 2021} \\
			
			\hspace{-4ex}{2015 - 2019} & {\bf Shanghai Jiao Tong University}, Shanghai, China\\
			& {~B.S. Aeronautics and Astronautics Engineering},  June 2019 (Average Score: 89.22/100)\\[-1ex]
		\end{longtable}
		\vsrk\vsrk
		\section{EMPLOYMENT}
		\sepline
		\begin{longtable}{@{}>{\raggedright}p{0.07\linewidth}
				p{\dimexpr0.95\linewidth-2\tabcolsep-\arrayrulewidth\relax}@{}}
			\hspace{-4ex}{2020 - } & {\bf Research Assistant}, Bio-Inspired Motion Lab at USC, PI: \textit{Prof.~Eva Kanso}\\
			\hspace{-4ex}{2019} & {\bf Intern Algorithmic Engineer}, Shanghai Hongpu Information Technology Co., Ltd. \\
			\hspace{-4ex}{ } & Conduct flaw detection on images of photovoltaic cell using Faster R-CNN and yolov3 \\
			\hspace{-4ex}{2016 - 2019} & {\bf Research Assistant}, J.C.Wu Center for Aerodynamics, PI: \textit{Prof.~Hong Liu}\\
		\end{longtable}
		\vsrk\vsrk
		\section{PUBLICATIONS}
		\sepline
		\begin{longtable}{@{}>{\raggedright}p{0.07\linewidth}
				p{\dimexpr0.95\linewidth-2\tabcolsep-\arrayrulewidth\relax}@{}}
			\hspace{-4ex}{2022} 
			& \nhphantom{3. }3. Hang, H., Heydari, S., Costello, J., \& Kanso, E. (2022). \href{https://doi.org/doi:10.1017/jfm.2021.984}{\textit{Active tail flexion in concert with passive hydrodynamic forces improves swimming speed and efficiency}}. Journal of Fluid Mechanics, 932, A35.  \\
			\hspace{-4ex}{2021} 
			& \nhphantom{2. }2. Xiang, Y., Hang, H., Qin, S., and Liu, H. (2021). \href{https://doi.org/10.1007/s10409-021-01134-7}{\textit{Scaling analysis of the circulation growth of leading-edge vortex in flapping flight}}. Acta Mech. Sin.  \\
			\hspace{-4ex}{2020} & \nhphantom{1. }1. Hang, H., Yu, B., Xiang, Y., Zhang, B., and Liu, H. (2020).  \href{https://doi.org/10.1007/s12650-019-00605-1}{\textit{An objective-adaptive refinement criterion based on modified ridge extraction method for finite-time Lyapunov exponent (FTLE) calculation}}. Journal of Visualization, 23(1), 81-95. \\
		\end{longtable}
		\vsrk\vsrk
		\section{RESEARCH INTERESTS/EXPERIENCE}
		\sepline
		\begin{longtable}{@{}>{\raggedright}p{0.07\linewidth}
				p{\dimexpr0.95\linewidth-2\tabcolsep-\arrayrulewidth\relax}@{}}
			
			\hspace{-4ex}{2020 - } & {\bf Learning to blindly follow hydrodynamic trails}, supervised by \textit{Prof.~Eva Kanso} \\
			& \hspace{2ex} joint with \textit{Sina Heydari}, \textit{Yusheng Jiao}, \textit{Feng Ling}\\
			& Employ reinforcement learning to follow vortical wakes based on local flow sensory\\
			& Find traveling wave characteristic of wakes is important for source seeking and our controller is stable in locating source \\
			& Compare performance between mechano- and chemo- sensing, and different sensory cues\\
		\end{longtable}
		\vspace*{-4.5ex}
		\begin{longtable}{@{}>{\raggedright}p{0.07\linewidth}
				p{\dimexpr0.95\linewidth-2\tabcolsep-\arrayrulewidth\relax}@{}}
			
			\hspace{-4ex}{2020 - } & {\bf Flexion in fish swimming}, supervised by \textit{Prof.~Eva Kanso}, \textit{Prof.~John H. Costello}\\
			& \hspace{2ex} joint with \textit{ Sina Heydari}\\
			& Analyze the role of active and passive flexion on swimming speed and efficiency of a self-propelling pitching plate using vortex sheet method \\
			& Parametric study on effects of flexion phase, flexion angle and flexion ratio on swimming performance\\

						
		\end{longtable}
		\vspace*{-4.5ex}
		\begin{longtable}{@{}>{\raggedright}p{0.07\linewidth}
				p{\dimexpr0.95\linewidth-2\tabcolsep-\arrayrulewidth\relax}@{}}
			\hspace{-4ex}{2016 - 2019} & {\bf Mechanisms of high generation in insects flight}, supervised by \textit{Prof.~Hong Liu} \textit{Dr.~Yang Xiang} and \textit{Dr.~Suyang Qing}\\
			& Conduct experimental study using robotic flapping wing models in glycerin with Reynolds number similar to insects \\
			& Measure flow field using particle image velocimetry (PIV) and measure force and torque using 6-axis force sensor, analogue filter and NI data acquisition system \\
			& Study difference in formation of leading edge vortex(LEV) for different kinematic modes, and find advanced rotation can generate a lager LEV because of wake capture \\
			& Find a scaling law between passive rotation and active translation in flapping wing model \\

		\end{longtable}
		\vspace*{-4.5ex}
		\begin{longtable}{@{}>{\raggedright}p{0.07\linewidth}
				p{\dimexpr0.95\linewidth-2\tabcolsep-\arrayrulewidth\relax}@{}}
			
			\hspace{-4ex}{2016 - 2019} & {\bf AMR for FTLE calculation}, supervised by \textit{Prof.~Hong Liu}, \textit{Prof.~Bin Zhang}, \textit{Bin Yu} and \textit{Dr.~Yang Xiang}\\
			& Develop a physics-based adaptive refinement method for finite-time Lyapunov exponent calculation\\
			
		\end{longtable}
		\vspace*{-4.5ex}
		\begin{longtable}{@{}>{\raggedright}p{0.07\linewidth}
				p{\dimexpr0.95\linewidth-2\tabcolsep-\arrayrulewidth\relax}@{}}
			
			\hspace{-4ex}{2015 - 2016} & {\bf VTOL pitch-changed quadrotor }, supported by National Students’ Platform for Innovation and Entrepreneurship Training , supervised by \textit{Prof.~Junqi Wu}\\
			& \hspace{2ex} joint with \textit{Dongming Ding}, \textit{Jihong Huang}, \textit{Chaoqun Li}, \textit{Zhikang Qiu}\\
			& Lead a team to make a quad rotor and fulfilled of pitch-changed technique and VTOL technique in terms of both mechanical and control\\
			% & Include flight test and numerical simulation \\
		\end{longtable}
		
		\vsrk\vsrk
		\section{TALKS/PRESENTATIONS}
		\sepline
		\begin{longtable}{@{}>{\raggedright}p{0.07\linewidth}
				p{\dimexpr0.95\linewidth-2\tabcolsep-\arrayrulewidth\relax}@{}}
			\hspace{-4ex}{2021} & {\bf APS Division of Fluid Dynamics Meeting}, Active tail flexion in concert with passive hydrodynamic forces improves swimming speed and efficiency\\
			\hspace{-4ex}{2020} & {\bf APS Division of Fluid Dynamics Meeting}, Flowtaxis in the wakes of oscillating airfoils\\
			\hspace{-4ex}{2018} & {\bf APS Division of Fluid Dynamics Meeting}, Passive rotation of a flapping wing with an inhomogeneous mass distribution \\

		\end{longtable}
%		\vsrk\vsrk


		\section{TEACHING EXPERIENCE}
		\sepline
		\begin{longtable}{@{}>{\raggedright}p{0.07\linewidth}
				p{\dimexpr0.95\linewidth-2\tabcolsep-\arrayrulewidth\relax}@{}}
			& \textbf{at University of Southern California} \\
			

			\hspace{-4ex}{2021 Spring} & {\bf Teaching Assistant}, AME-526, Introduction to mathematical methods in engineering II, \emph{Prof.~Niema Pahlevan}\\
	
			\hspace{-4ex}{2020 Fall} & {\bf Teaching Assistant}, AME-404, Computational Solutions to Engineering Problems, \emph{Prof.~Takahiro Sakai}\\
			
		\end{longtable}
		\vsrk\vsrk
		\section{SELECTED GRADUATE COURSEWORK}
		\sepline
		\begin{longtable}{@{}>{\raggedright}p{0.07\linewidth}
				p{\dimexpr0.95\linewidth-2\tabcolsep-\arrayrulewidth\relax}@{}}
			& \textbf{at University of Southern California} \\
			\hspace{-4ex}{2021} & PHYS-516, Methods of Computational Physics, A , \emph{Prof.~Aiichiro Nakano}\\
			& EE-556, Stochastic Systems and Reinforcement Learning, A , \emph{Prof.~Rahul Jain}\\
			% & CSCI-570, Analysis of Algorithms, B+ , \emph{Prof.~Victor Adamchik}\\
			& AME-508, Machine Learning and Computational Physics, A , \emph{Prof.~Assad Oberai}\\
			& CSCI-596, Scientific Computing and Visualization, A , \emph{Prof.~Aiichiro Nakano}\\
			& \hspace{4ex} \textbf{class project:} a C++ parallel reinforcement learning implementation \href{https://github.com/haotianh9/Reinforcement_learning_cpp_parallel}{\textit{Github Link}} \\
			\hspace{-4ex}{2020} %& AME-525, Engineering Analysis, A- ,\emph{Prof.~Eva Kanso}\\
			% & AME-526, Introduction to Mathematical Methods in Engineering II, A , \emph{Prof.~Niema Pahlevan}\\
			% & AME-511, Compressible Gas Dynamics, A , \emph{Prof.~Iván Bermejo-Moreno}\\
			% & PHYS-760, Selected Topics in Computational Physics, P , \emph{Prof.~Satish Kumar Thittamaranahalli}\\
			& AME-451, Linear Control Systems I, A , \emph{Prof.~Henryk Flashner}\\
			% & AME-541, Linear Control Systems II, A- , \emph{Prof.~Néstor O. Pérez-Arancibia}\\
			& AME-535A, Introduction to Computational Fluid Mechanics, A , \emph{Prof.~Alejandra Uranga}\\
			% & AME-530A, Dynamics of Incompressible Fluids, B+ , \emph{Prof.~Carlos Pantano}\\

		\end{longtable}
		\vsrk\vsrk

		\section{HONOR/AWARDS}
		\sepline
		\begin{longtable}{@{}>{\raggedright}p{0.07\linewidth}
				p{\dimexpr0.95\linewidth-2\tabcolsep-\arrayrulewidth\relax}@{}}
				
			\hspace{-4ex}{2017-2018} & Hui-Chun Chin and Tsung-Dao Lee Chinese Undergraduate Research Endowment of SJTU \\
			\hspace{-4ex}{2016} & Honeywell Star Project  \\
			& {\bf Second Place }, Parts of the National College Students Physics Competition \\
			& {\bf Third Place }, Chinese College Students' Mathematics Competition \\
			\hspace{-4ex}{2014} & {\bf First Place}, Chinese Chemistry Olympiad \\
			 & {\bf First Place}, Shanghai Adolescents Science and Technology Innovation Contest \\

		\end{longtable}
		\vsrk\vsrk
		\section{ONLINE COURSEWORK}
		\sepline
		\begin{longtable}{@{}>{\raggedright}p{0.07\linewidth}
				p{\dimexpr0.95\linewidth-2\tabcolsep-\arrayrulewidth\relax}@{}}
				& \textbf{at Udacity} \\
			\hspace{-4ex}{2022} & C++ Nanodegree \\
				& \textbf{at Coursera} \\
			\hspace{-4ex}{2021} & Build a Modern Computer from First Principles: From Nand to Tetris (Project-Centered Course), Hebrew University of Jerusalem \\
			\hspace{-4ex}{2019} & {\bf Specialization}, DeepLearning.AI TensorFlow Developer , DeepLearning.AI (containing 4 courses) \\
			& {\bf Specialization}, Deep Learning, DeepLearning.AI (containing 5 courses) \\
			& Machine Learning, Stanford University, \\
			\hspace{-4ex}{2014} & General Chemistry, Peking University \\
		\end{longtable}
		\vsrk\vsrk
		
		\section{TECHNICAL SKILLS}
		\sepline
		\begin{longtable}{@{}>{\raggedright}p{0.07\linewidth}
				p{\dimexpr0.65\linewidth-2\tabcolsep-\arrayrulewidth\relax}@{}}
			\hspace{-24ex}{Programming Language: } & Python, \, Matlab, \, Fortran, \, C/C++  (from more familiar to less, same below) \\
			\hspace{-24ex}{Machine learning framework: } & Pytorch, Tensorflow \\
			\hspace{-24ex}{Hardware: } & Arduino, Raspberry Pi, Pixhawk \\
		\end{longtable}
		\vsrk\vsrk
	\end{resume} 
\end{document}